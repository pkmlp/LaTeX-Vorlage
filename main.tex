
\documentclass[12pt, a4paper, parskip=full]{scrartcl}

% Die Gestaltung der Absätze ist im deutschen Sprachraum   
% anders als im englischen Sprachraum. Die Anpassung an
% den deutschen Sprachraum erreichen Sie mit: parskip=full.

% Für eine vollständige Anpassung an die deutsche Sprache;
\usepackage[utf8]{inputenc}   % für deutsche Umlaute
\usepackage[ngerman]{babel}   % für neue deutsche Rechtschreibung 
\usepackage[T1]{fontenc}      % für Umlaute im pdf-Dokument

% Angaben für die automatische Titelseite
\title{\LaTeX-Basics}         % Titel der Arbeit
\author{Peter Kessler}        % Autor der Arbeit
\date{\today}                 % Datum der Arbeit

% Packages für Mathe
\usepackage{amsmath}          % für mathematische Formeln
\usepackage{amssymb}          % für mathematische Symbole

% Packages für Inhaltsverzeichnis mit Links
\usepackage{hyperref}
\hypersetup{
    colorlinks,              % wenn Links farbig markiert werden sollen
    citecolor=black,         % keine farbige Unterscheidung bei Querverweisen
    linkcolor=black,         % keine farbige Unterscheidung der ToC Zeilen
    urlcolor=black           % keine farbige Unterscheidung bei URL-Links
}

% Packages für Bilder und Grafiken
\usepackage{graphicx}         % für Bilder und Grafiken

% Package für Literaturverzeichnis und Zitierungen
\usepackage{csquotes}

% für Literatur-Verweise im Stil [1]
\usepackage[style=numeric]{biblatex}            

% für Literatur-Verweise im Stil [Kot11]
%\usepackage[style=alphabetic]{biblatex}         

% Die Datei mit Literatur/Quellen Angaben
\addbibresource{./Verzeichnisse/Literatur.bib}  



\begin{document}

% Titelseite erstellen mit Unterdrückung der Seitenzahl auf der Titelseite
\maketitle
\thispagestyle{empty}   


% Inhaltsverzeichnis auf neuer Seite
\pagebreak
\tableofcontents


% Kapitel auf neuer Seite
\pagebreak 
\section{Herkunft}
Das Basis-Programm von LaTeX ist TeX und wurde von Donald E. Knuth während seiner Zeit als Informatik-Professor an der Stanford University entwickelt. Auf TeX aufbauend entwickelte Leslie Lamport\cite{Lamport2017} Anfang der 1980er Jahre LaTeX, eine Sammlung von TeX-Makros, die die Benutzung für den durchschnittlichen Anwender gegenüber TeX vereinfachten und erweiterten. Der Name LaTeX ist eine Abkürzung für Lamport TeX.

\begin{figure}[h!]
\centering
  \includegraphics[width=0.5\textwidth]{./Bilder/overleaf.png}
  \caption{Overleaf - \LaTeX in der Cloud}
\end{figure}

Ein sehr zu empfehlendes Buch zum Thema \LaTeX\ ist\cite{Kottwitz2011}. 



% Kapitel auf neuer Seite
\pagebreak
\section{WYSIWYG vs WYSIWYAF}

\subsection{WYSIWYG}
Im Gegensatz zu anderen Textverarbeitungsprogrammen, die nach dem WYSIWYG (What-you-see-is-what-you-get) Prinzip funktionieren, arbeitet der Autor mit Textdateien, in denen er innerhalb eines Textes anders zu formatierende Passagen oder Überschriften mit Befehlen textuell auszeichnet.

\subsection{WYSIWYAF}
Das dabei von LaTeX generierte Layout gilt als sehr sauber, sein Formelsatz als sehr ausgereift. Außerdem ist die Ausgabe u. a. nach PDF, HTML und PostScript möglich. LaTeX eignet sich insbesondere für umfangreiche Arbeiten wie Diplomarbeiten und Dissertationen, die oftmals strengen typographischen Ansprüchen genügen müssen. Insbesondere in der Mathematik und den Naturwissenschaften erleichtert LaTeX das Anfertigen von Schriftstücken durch seine komfortablen Möglichkeiten der Formelsetzung gegenüber üblichen Textverarbeitungssystemen. Das Verfahren von LaTeX wird auch mit WYSIWYAF (What you see is what you asked for.) umschrieben.

Wie TeX selbst ist LaTeX weitestgehend rechnerunabhängig verwendbar. Das bedeutet, dass es für die meisten Betriebssysteme analog zu TeX auch für LaTeX lauffähige, produktiv einsetzbare LaTeX-Installationen gibt. Zu diesen Betriebssystemen gehören zum Beispiel Microsoft Windows von der Version 3.x bis zur aktuellen Version 10, Apple macOS sowie diverse Linux-Distributionen. Unter der Voraussetzung, dass alle verwendeten Zusatzpakete (siehe unten) in geeigneten Versionen installiert sind, besteht der Vorteil der Verwendung von LaTeX darin, dass das Ergebnis unabhängig von der verwendeten Rechnerplattform und dem verwendeten Drucker in den beiden Ausgabeformaten DVI und PDF hinsichtlich der Schriftarten und -größen sowie der Zeilen- und Seitenumbrüche im Druck stets exakt gleich ist. LaTeX ist hierbei nicht auf die Schriftarten des jeweiligen Betriebssystems angewiesen. Jene Betriebssystem-Schriftarten sind häufig für die Anzeige am Monitor optimiert. LaTeX enthält eine Reihe eigener Schriftarten, die ihrerseits für den Druck optimiert sind. 



% Kapitel auf neuer Seite
\pagebreak
\section{Arbeiten mit mathematischen Formeln}
LaTeX unterstützt wie kaum ein anderes Textsatzungsprogramm das Einfügen mathematischer Formeln. Wahlweise können Formeln innerhalb einer Absatzes („Inline“) oder als eigenständiger Absatz in ein Dokument eingefügt werden.



\subsection{Inline Formeln}
Pythagoras sagt: Seien $a$ und $b$ die 
Katheten und $c$ die Hypotenuse, dann gilt: $a^2+b^2=c^2$. 
Somit gilt für die Hypothenuse: $c=\sqrt{a^2+b^2}$.



\subsection{Abgesetzte Formeln}

Pythagoras sagt: Seien $a$ und $b$ die Katheten und $c$ die
Hypotenuse, dann gilt: 

\begin{displaymath}
  a^2+b^2=c^2 
\end{displaymath}
 
Somit gilt für die Hypothenuse: 

\begin{displaymath}
  c=\sqrt{a^2+b^2} 
\end{displaymath}


\subsection{Ausgerichtete Formeln}

Da gibt es zwei Formeln, die jedes Kind kennt: 
\begin{align}
  a^2 + b^2 &= c^2     \label{pythagoras}  \\  
  e &= m c^2           \label{einstein}  
\end{align}

Wobei \eqref{pythagoras} Pythagoras und \eqref{einstein} Albert Einstein zugeschrieben wird.


% Kapitel auf neuer Seite
\pagebreak
\section{Zitieren und Literaturverzeichnis}

Quellenangaben und Zitate erfüllen in wissenschaftlichen Texten bzw. in wissenschaftlichen Arbeiten zwei Hauptziele\cite{NW2017}: 

\begin{itemize}

\item[a)] Nachvollziehbarkeit der Argumentation, des (gedanklichen) Experiments
Ein wissenschaftlicher Beitrag kann nur dann weiterverwendet werden, wenn sich die Argumentation, das (gedankliche) Experiment für die Lesenden überprüfen lässt.

\item[b)] Unterscheidung zwischen eigenen und fremden Gedanken (geistiges Eigentum)
Ideen, Beispiele, ein bestimmtes methodisches Vorgehen, Bilder, Grafiken, Tabellen u.a., die aus anderen Texten stammen, müssen als fremdes geistiges Eigentum ausgewiesen werden.

\end{itemize}

\textbf{Wichtige Regel:} Quellenangaben gehören zum Satz dazu und stehen daher VOR dem Punkt bzw. Satzzeichen.


% Literatur- und Quellenverzeichnis auf neuer Seite
\pagebreak

% Eintrag für Literatur und Quellenverzeichnis in ToC erstellen
\addcontentsline{toc}{section}{Literatur- \& Quellenverzeichnis}    

% Literatur- und Quellenverzeichnis erstellen
\printbibliography[title={Literatur- \& Quellenverzeichnis}]        

\end{document}
